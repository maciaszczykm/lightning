\documentclass[12pt]{report}

\usepackage[T1]{fontenc}
\usepackage[utf8]{inputenc}
\usepackage{graphicx}
\usepackage{amsmath,amssymb,amsfonts}
\usepackage{txfonts}
\usepackage{pdfpages}
\usepackage{caption}
\usepackage{float}
\usepackage{listings}
\usepackage[polish]{babel}

\renewcommand{\chaptername}{Rozdział}
\renewcommand{\contentsname}{Spis treści}
\renewcommand{\figurename}{Rys.}
\renewcommand{\tablename}{Tab.}
\renewcommand{\listfigurename}{Spis rysunków}
\renewcommand{\listtablename}{Spis tabel}
\renewcommand{\bibname}{Bibliografia}
\renewcommand\lstlistingname{Listing}
\renewcommand\lstlistlistingname{Spis listingów}

\pagestyle{headings}

\setlength{\textwidth}{14cm}
\setlength{\textheight}{20cm}

\newtheorem{definition}{Definicja}
\newtheorem{example}{Przykład}[chapter]
\newtheorem{corollary}{Wniosek}[chapter]

\begin{document}

\lstset{aboveskip=20pt,belowskip=20pt}

\lstdefinestyle{customcmd}{
captionpos=b, 
belowcaptionskip=1\baselineskip,
breaklines=true,
frame=LRTB,
xleftmargin=\parindent,
showstringspaces=false,
basicstyle=\footnotesize\ttfamily
}

\lstdefinestyle{customswift}{
captionpos=b, 
belowcaptionskip=1\baselineskip,
breaklines=true,
frame=LRTB,
xleftmargin=\parindent,
language=Swift,
showstringspaces=false,
basicstyle=\footnotesize\ttfamily,
keywordstyle=\bfseries\color{green!40!black},
commentstyle=\itshape\color{purple!40!black},
identifierstyle=\color{blue},
stringstyle=\color{orange},
}

% TODO strona tytułowa

\tableofcontents

\chapter{Wstęp}

\section{Uzasadnienie wyboru tematu}

Wraz z upływem czasu postęp technologiczny ma wpływ na życia co raz szerszej rzeszy ludzi na całym świecie. Niezliczone ilości urządzeń zagościły na stałe w domach i mało kto wyobraża sobie bez nich swoje życie. Zaczynając od artykułów gospodarstwa domowego, a kończąc na elektronice użytkowej do której zaliczają się komputery, telewizory czy też smartfony\footnote{~Przenośne urządzenia łączące w sobie zalety telefonów komórkowych oraz przenośnych komputerów (z ang. smartphone).}. Wszystkie te urządzenia mają na celu ułatwianie życia swoim użytkownikom.

W parze z licznymi zaletami urządzeń elektronicznych idą jednak pewne wady. Jedną z istotniejszych jest wpływ czasu spędzanego przed różnego rodzaju wyświetlaczami na zdrowie. Badania przeprowadzone na bazie danych Nielsen Audience Measurement pokazują, że przeciętny Polak spędza dziennie średnio 4,5 godziny przed ekranem telewizora\footnote{~Badania zostały przeprowadzone z uwzględnienem osób powyżej 4 roku życia w okresie od stycznia do czerwca 2015 roku \cite{czasprzedtv}.}. Nie oznacza to jednak, że przez cały ten czas ogląda on telewizję. Oglądanie filmów z dysku komputera, za pomocą serwisów VOD\footnote{~Wideo na życzenie (z ang. video on demand).} czy granie na konsoli także są wliczone w ten czas. Gdyby jednak dodać do tego czas spędzony przed ekranem smartfona czy też komputera wynik byłby zapewne dwukrotnie większy.

% TODO wsparcie co najmniej dwoma źródłami i coś o oglądaniu w ciemności
Pogorszenie wzroku czy też wysychanie gałki ocznej są wymieniane jako naj\-częstsze skutki zbyt dużej ilości czasu spędzanego przed ekranem. Poza próbą jego ograniczenia, jedną z częstszych porad jest próba zmniejszenia kontrastu pomiędzy ekranem a jego otoczeniem.

Do celów niniejszej pracy należy złożenie i oprogramowanie systemu oświetlenia, który ma rozszerzać obraz widziany na ekranie na jego otoczenie. Poza zmniejszeniem kontrastu, a więc aspektem zdrowotnym, system ma także na celu zwiększyć wrażenia wizualne dostarczane przez oglądany obraz.

% TODO krótkie wyjaśnienie podstawowych pojęć i diagram/zdjęcia
System oświetlenia składa się z taśmy diod elektroluminescencyjnych\footnote{~LED (z ang. light-emitting diode).} pod\-łączonych do mikrokontrolera Arduino Uno, który z kolei ma współpracować z komputerem z zainstalowanym systemem operacyjnym MacOS. Oprogramowanie mikrokontrolera, którego zadaniem jest sterowanie diodami zostało przygotowane w języku Arduino, natomiast aplikacja kontrolująca cały system przeznaczona na komputer z systemem MacOS została napisana w języku Swift. Wybór języków jest ściśle związany z koniecznością uzyskania jak najlepszej wydajności oraz użyciem najnowszych technologii.

Podobne systemy oświetlenia dostępne są już od pewnego czasu na rynku, jednak to właśnie nowoczesne technologie, prostota wykonania i niskie koszta powinny uczynić z Ligtning, bo taką nazwę otrzymał projekt, pełnowartościowego konkurenta.

\section{Problematyka i zakres pracy}

\section{Cele pracy}

\section{Metoda badawcza}

\section{Przegląd literatury w dziedzinie}

\section{Układ pracy}

\chapter[Zagadnienia teoretyczne]{Zagadnienia teoretyczne}

\chapter[Analiza istniejących rozwiązań]{Analiza istniejących rozwiązań}

\section{Kryteria analizy}

\section{Porównanie istniejących rozwiązań}

\chapter{System oświetlenia Lightning}

\section{Komponenty systemu}

\section{Moduły oprogramowania}

\subsection{Oprogramowanie przeznaczone na mikrokontroler}

\subsection{Aplikacja przeznaczona na komputer}

\section{Analiza wymagań}

\subsection{Wymagania funkcjonalne}

W celu stworzenia jak najatrakcyjniejszego system oświetlenia podczas projektowania Lightning przyjęto poniższe wymagania funkcjonalne:

\begin{itemize}
\item System musi udostępniać tryb przechwytywania rozszerzający obraz wyświetlany na ekranie na jego otoczenie za pomocą diod elektroluminescencyjnych.
\item System musi posiadać tryb animacji za pomocą diod elektroluminescencyjnych. Powinien być on łatwy do rozszerzenia o kolejne animacje.
\item System musi być konfigurowalny z poziomu aplikacji sterującej. Konfiguracji podlegać musi co najmniej liczba i rozmieszczenie diod za ekranem, a także wykorzystywany port szeregowy.
\item Projekt musi być odpowiednio udokumentowany. Dokumentacja powinna ułatwiać szybkie skonfigurowanie oraz uruchomienie systemu.
\end{itemize}

\subsection{Wymagania niefunkcjonalne}

Do wymagań niefunkcjonalnych postawionych przed projektowanym systemem należą:

\begin{itemize}
\item System musi działać płynnie, a więc liczba osiąganych klatek na sekundę powinna być jak największa nawet przy dużych rozdzielczościach przechwytywanego ekranu.
\item Korzystanie z aplikacji sterującej powinno być jak najbardziej intuicyjne, a użytkownik powinien mieć do wskazówek dotyczących jej interfejsu.
\item Oprogramowanie mikrokontrolera powinno oferować jak najprostszy interfejs i zawierać jak najmniej logiki sterującej.
\item Pamięć na obu urządzeniach sterujących powinna być odpowiednio zarządzana, niedopuszczalne są wycieki pamięci czy też zapętlenia programu.
\end{itemize}

\section{Projekt}

\begin{figure}[h]
\centering
\includegraphics[width=.85\textwidth]{../resource/logo.png}
\caption{Logo projektu}
\end{figure}

\section{Implementacja}

\section{Podręcznik użytkownika}

\section{Przykładowa implementacja animacji}

\section{Analiza projektu}

\section{Perspektywy rozwoju projektu}

\chapter{Podsumowanie}

\section{Dyskusja wyników}

\section{Perspektywy rozwoju pracy}

\addcontentsline{toc}{chapter}{Bibliografia} 
\begin{thebibliography}{99}
\bibitem{czasprzedtv} {\tt http://www.wirtualnemedia.pl/artykul/coraz-dluzej-ogladamy\--telewizje-najwiecej-czasu-przed-szklanym-ekranem-spedzaja\--seniorzy-raport}. Data dostępu -- 15.01.2017.
\end{thebibliography}

\addcontentsline{toc}{chapter}{Spis rysunków} 
\listoffigures

\addcontentsline{toc}{chapter}{Spis tabel} 
\listoftables

\addcontentsline{toc}{chapter}{Spis listingów} 
\lstlistoflistings

\end{document}